\documentclass[]{article}
   \usepackage{nopageno}
%opening
\title{UPA  - Ukládání a příprava dat \\ Projekt 1. část: návrh zpracování a uložení dat}

\date{\vspace{-9ex}}
%\date{}  % Toggle commenting to test
%\author{Slávka Samuel\\ \texttt{xslavk02} \and Šajdík Ondrej\\ \texttt{xsajdi01} \and Šalgovič Marek\\ \texttt{xsalgo00}}
\setlength{\voffset}{-0.75in}
\begin{document}


\maketitle


%\centerline{\rule{13cm}{0.4pt}}

\paragraph{Zvolená téma:}

03: Kurzy devizového trhu 
\paragraph{Riešitelia:}
\begin{itemize}
\item Samuel Slávka - \texttt{xslavk02} \item Ondrej Šajdík - \texttt{xsajdi01} \item Marek Šalgovič - \texttt{xsalgo00}
\end{itemize}

 \paragraph{Zvolené dotazy a formulácie vlastného dotazu:}

\begin{itemize}
\item \textbf{Dotaz skupiny A} - vytvořte žebříček měn, které v daném období nejvíce posílily/oslabily
\item \textbf{Dotaz skupiny B} - najděte skupiny měn s podobným chováním (skupiny měn, které obvykle současně posilují/oslabují)
\item \textbf{Vlastný dotaz} - nájdenie najstabilnejšej meny počas určitého obdobia
\end{itemize}

Vstupom vlastného dotazu sú dva vyhovujúce dátumy. Dotazom sa nájde mena, ktorá bola medzi týmito dvomi dňami najstabilnejšia. Stabilita meny je vyhodnotená ako súčet odchyliek v jednotlivých dňoch od priemernej hodnoty v tomto období. Nejedná sa teda iba o rozdiel hodnôt meny v začiatočnom a koncovom čase.

\paragraph{Stručná charakteristika zvolenej dátovej sady:}

Zadaná dátová sada obsahuje viacero súborov vhodných pre získanie potrebných dát. Zdroj ponúka získanie súboru s dátami pre konkrétny deň na adrese: 
\begin{center}
 https://www.cnb.cz/cs/financni-trhy/devizovy-trh/kurzy-devizoveho-trhu/kurzy-devizoveho-trhu/denni\_kurz.txt?date=DATE
 \end{center}
  kde zvolený URL query parameter DATE určuje dáta pre zvolený deň.
  
   Pre množstvo nepotrebných informácií pre naše dotazy, ako napríklad meno krajiny alebo celý názov meny, sme sa rozhodli použiť súbor s dátami za celý zvolený rok. Adresa pre ročné dáta: 
\begin{center}
https://www.cnb.cz/cs/financni-trhy/devizovy-trh/kurzy-devizoveho-trhu/kurzy-devizoveho-trhu/rok.txt?rok=YEAR
\end{center}
kde URL parameter YEAR určuje rok.

V ročnom súbore sú dáta uložené v tabuľkovej schéme. Prvý stĺpec tabuľky obsahuje dátum, pre ktorý platia kurzy v riadku. V záhlaví tabuľky sú konkrétne skratky mien spolu s ich množstvom. 

\pagebreak

Príklad tabuľky:

\begin{center}
\begin{tabular}{ |c|c|c| } 
 \hline
 Dátum & X mena1 & Y mena2 \\ 
  \hline
 1. 1. 2020  & w & x \\ 
  \hline
 2. 1. 2020 & y & z \\ 
 \hline
\end{tabular}
\end{center}

kde X,Y je množstvo meny a w, x, y, z sú konkrétne kurzy. (pozn.: celkový kurz meny je rovný pomeru kurz/množstvo).

Dáta budú spracované stiahnutím tohto súboru a následnom parsovaním tabuľky a uložením do NoSQL databázy.

\paragraph{Spôsob získania a spracovania dat:} Dáta sme získali pomocou \texttt{HTTP GET} dotazu v jazyku \textit{Golang}.  Parsovanie výsledku je riešené rozdelením textového dokumentu podľa znaku '\textbar', ktorý oddeľuje jednotlivé buňky v tabuľke. Výsledok sa uložil do dátovej štruktúry skladajúcej sa z dátumu a kolekcie dvojíc - skratka názvu meny a jej hodnota pre daný dátum. Hodnota je vyjadrená celkovým kurzom menu, ako sme zadefinovali vyššie.


\paragraph{Zvolený spôsob uloženía surových dat:}

Na uloženie dát zo zdrojových súborov sme sa rozhodli použiť NoSQL databázu typu dokumentovej databázy. Dokumentová databáza umožňuje ukladať dokumenty do kolekcií. Tieto dokumenty sú často reprezentované štruktúrou \texttt{JSON}. V kolekcii, čo je vlastne súbor dokumentov, sa môžu nachádzať aj dokumenty s rozdielnymi schémami. V našom prípade nám to umožňuje uložiť aj novo pridané meny do zdrojových súborov.  Príkladom konkrétnej dokumentovej NoSQL databázy je \textit{mongoDB}. 

Pre uloženie dát sme navrhli kolekciu \textit{days}, ktorej jednotlivé dokumenty reprezentujú stav každej meny v deň podľa zdrojových súborov. 


\end{document}
